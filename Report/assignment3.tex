\section*{Assignment 3}\label{sec:Ass3}
In Assignment 3 we employ the simulation set-up developed in Assignments 1 and 2 to investigate
the effects of different key process parameters on the efficiency of the wastewater treatment plant.
To carry-out this analysis please use a constant Kla value for tank5 for the nominal case as given in
section 4 of the BSM1 report. For the influent data utilise the file ”In f dry.txt” for ASM1 initialised at
steady-state, which may be different for each variation of the process parameters.
\newline
Gauge the effect of the following parameters on the plant using the performance indices developed
previously, explain your observations and state the advantages and disadvantages:

\subsection*{(a)}
\textit{Implement the remaining ME quality indicator over the full 14 days and combine all quality
indicators for the overall cost indicator (OCI).For ME use the ”i f ”else expression in Modelica.} \newline
We implementet a new sensor for measuring OCI using the formula from the benchmarkWWTP: \newline
OCI = AE + PE +5*SP + 3*EC + ME

\subsection*{(b)}
\textit{Volumetric flow-rates of the pumps.} \newline
Change in volumetric flow-rates of the pumps does not affect the quality indicators.

\subsection*{(c)}
\textit{Kla values of the aeration tanks} \newline
When Kla for tank 3 and 4 is reduced by 50\% from 240 to 120, AE is reduced from 3700 to 2250.

\subsection*{(d)}
\textit{$Y_A$, $Y_H$ and $SO_{sat}$ (for these cases ignore advantages and disadvantages)} \newline
Change in $SO_{sat}$ from 8 to 4 does not affect the quality indicators. A change in Y\_a from 0.67 to 0.02 and Y\_h from 0.24 to 0.02 does not affect the quality indicators.
